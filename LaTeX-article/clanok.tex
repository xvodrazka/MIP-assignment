% Metódy inžinierskej práce

\documentclass[10pt,twoside,slovak,a4paper]{article}

\usepackage[english]{babel}
%\usepackage[T1]{fontenc}
\usepackage[IL2]{fontenc} % lepšia sadzba písmena Ľ než v T1
\usepackage[utf8]{inputenc}
\usepackage{graphicx}
\usepackage{url} % príkaz \url na formátovanie URL
\usepackage{hyperref} % odkazy v texte budú aktívne (pri niektorých triedach dokumentov spôsobuje posun textu)

\usepackage{cite}
%\usepackage{times}

\pagestyle{headings}

\title{Improving performance in the workplace using gamification methods\thanks{Semestrálny projekt v predmete Metódy inžinierskej práce, ak. rok 2022/23, vedenie: Mirwais Ahmadzai}} % meno a priezvisko vyučujúceho na cvičeniach

\author{Júda Vodrážka\\[2pt]
	{\small Slovenská technická univerzita v Bratislave}\\
	{\small Fakulta informatiky a informačných technológií}\\
	{\small \texttt{xvodrazka@stuba.sk}}}

\date{\small 25. October 2022} % upravte



\begin{document}

\maketitle

\begin{abstract}

\end{abstract}





\section{Úvod}

Motivujte čitateľa a vysvetlite, o čom píšete. Úvod sa väčšinou nedelí na časti.

Uveďte explicitne štruktúru článku. Tu je nejaký príklad.
Základný problém, ktorý bol naznačený v úvode, je podrobnejšie vysvetlený v časti~\ref{nejaka}.
Dôležité súvislosti sú uvedené v častiach~\ref{dolezita} a~\ref{dolezitejsia}.
Záverečné poznámky prináša časť~\ref{zaver}.



\section{Nejaká časť} \label{nejaka}

Z obr.~\ref{f:rozhod} je všetko jasné. 

\begin{figure*}[tbh]
\centering
%\includegraphics[scale=1.0]{diagram.pdf}
Aj text môže byť prezentovaný ako obrázok. Stane sa z neho označný plávajúci objekt. Po vytvorení diagramu zrušte znak \texttt{\%} pred príkazom \verb|\includegraphics| označte tento riadok ako komentár (tiež pomocou znaku \texttt{\%}).
\caption{Rozhodujúci argument.}
\label{f:rozhod}
\end{figure*}



\section{Iná časť} \label{ina}

Základným problémom je teda\ldots{} Najprv sa pozrieme na nejaké vysvetlenie (časť~\ref{ina:nejake}), a potom na ešte nejaké (časť~\ref{ina:nejake}).\footnote{Niekedy môžete potrebovať aj poznámku pod čiarou.}

Môže sa zdať, že problém vlastne nejestvuje\cite{Coplien:MPD}, ale bolo dokázané, že to tak nie je~\cite{Czarnecki:Staged, Czarnecki:Progress}. Napriek tomu, aj dnes na webe narazíme na všelijaké pochybné názory\cite{PLP-Framework}. Dôležité veci možno \emph{zdôrazniť kurzívou}.


\subsection{Nejaké vysvetlenie} \label{ina:nejake}

Niekedy treba uviesť zoznam:

\begin{itemize}
\item jedna vec
\item druhá vec
	\begin{itemize}
	\item x
	\item y
	\item z
	\end{itemize}
\end{itemize}

Ten istý zoznam, len číslovaný:

\begin{enumerate}
\item jedna vec
\item druhá vec
	\begin{enumerate}
	\item x
	\item y
	\item z
	\end{enumerate}
\end{enumerate}


\subsection{Ešte nejaké vysvetlenie} \label{ina:este}

\paragraph{Veľmi dôležitá poznámka.}
Niekedy je potrebné nadpisom označiť odsek. Text pokračuje hneď za nadpisom.



\section{Dôležitá časť} \label{dolezita}




\section{Ešte dôležitejšia časť} \label{dolezitejsia}


\section{cv02 section creation}


Lorem ipsum dolor sit amet, consectetur adipiscing elit. Donec dapibus eu enim at egestas. Aliquam et lectus ac purus commodo mattis. Phasellus et dapibus lectus. Donec ornare ipsum commodo, blandit magna blandit, pulvinar ligula. Vivamus rhoncus, ipsum vitae maximus lobortis, mi tellus placerat nulla, in tempor ex nisi id turpis. Nam id dolor semper, placerat ante sit amet, tincidunt neque. Fusce sagittis vehicula felis, eget consectetur neque hendrerit non. Nam blandit velit elit. Nunc non metus quis mi sagittis porta nec ut tortor. Nunc sit amet turpis tincidunt, vehicula purus quis, aliquet lacus. Duis quis mauris vitae arcu ultrices pellentesque. In ut mauris egestas dolor placerat efficitur.

Aliquam a augue accumsan, fringilla odio ut, vestibulum ante. Maecenas vel nunc luctus, ullamcorper ipsum id, viverra arcu. Suspendisse potenti. Suspendisse facilisis mi eget fringilla volutpat. Morbi fermentum est dui, nec pellentesque nisi finibus at. Donec erat nunc, faucibus sed nunc eu, varius facilisis erat. Mauris nec neque in purus scelerisque tempus. Pellentesque sodales faucibus nisl eget bibendum. Sed molestie augue at orci maximus, vel posuere massa suscipit. Mauris euismod tincidunt ornare.

Cras mattis enim et efficitur iaculis. Donec sit amet finibus diam. Quisque auctor porta massa, et varius urna vehicula vitae. Duis a lorem a nulla bibendum dictum. Praesent lacinia mattis urna vitae convallis. Duis molestie arcu id libero iaculis luctus. Aliquam sagittis libero ac dolor venenatis, id aliquam nibh consectetur. Nam feugiat, massa quis efficitur sagittis, leo ex faucibus tortor, a molestie dui purus ultrices massa. Nunc quis malesuada purus.

Praesent nunc sem, pharetra eget consequat quis, porttitor sed magna. Curabitur fringilla vel mauris et luctus. Praesent porttitor lectus lobortis ultricies eleifend. Nunc ac tempus nunc. Aliquam interdum velit non magna gravida laoreet. Phasellus convallis ligula ac laoreet feugiat. Mauris tellus eros, fringilla egestas libero non, vulputate condimentum nunc. In ullamcorper nisi nec interdum placerat. Nullam ac venenatis velit. Praesent tristique malesuada suscipit.




\section{Záver} \label{zaver} % prípadne iný variant názvu





%\acknowledgement{Ak niekomu chcete poďakovať\ldots}


% týmto sa generuje zoznam literatúry z obsahu súboru literatura.bib podľa toho, na čo sa v článku odkazujete
\bibliography{literatura.bib}
\bibliographystyle{abbrv} % prípadne alpha, abbrv alebo hociktorý iný
\end{document}
